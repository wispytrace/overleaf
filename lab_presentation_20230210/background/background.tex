%\usepackage[width=2cm,dark,tab]{beamerthemesidebar}
%\usepackage{colortbl} % ²ÊÉ«±í¸ñ
%\usepackage[table]{xcolor}
%\usepackage{booktabs} % ¿ÉÒÔʹÓÃÈýÏß±í
%\usepackage{multirow} % ¸´ÔÓ±í¸ñ£¬Ê¹ÓÃmultirow±ØÐë¼ÓÔظÃpackage
%\usepackage{dcolumn}
\usepackage{graphicx,subfigure,boxedminipage}
\usepackage{bm}
\usepackage{amsmath}
\usepackage[english]{babel}
% or whatever
\usepackage{CJK}
\usepackage[latin1]{inputenc}
% or whatever
\usepackage{alltt}
\usepackage{amsmath,amssymb,amsfonts}
\usepackage{times}
\usepackage{multimedia}
\usepackage{color}% [usenames]
\usepackage{hyperref}
%\usepackage[T1]{fontenc}
\graphicspath {{figure/}}%ͼƬËùÔÚµÄĿ¼
\usepackage{multicol}    %ͬʱʹÓõ¥ÁкͶàÁУ¬ÈçÏÂ
\usepackage{multirow}
%%%\begin{multicols}{2}
%%%\end{multicols}
\usepackage{setspace} % µ÷Õû¼ä¾à
\usepackage{boiboites} % ×Ô¶¨Ò嶨Àí»·¾³
\usepackage{textpos} % ʹlogo²»±»¸²¸Ç
%\begin{spacing}{1.5}
%\tableofcontents \listoffigures \listoftables
%\end{spacing}

\setlength{\columnsep}{-0.05cm} %Ë«À¸Ö®¼äµÄ¼ä¾à
%\setlength{\parskip}{-0.1\baselineskip} % ÉèÖöμä¾à

%%%%%%%%%%%%%%%%%%%%%%%%%%%%%%%%%%%%%%%%%%% Ä£°åģʽ¶¨Òå %%%%%%%%%%%%%%%%%%%%%%%%%%%%%%%%%%%%%%%%%%%
\mode<presentation> {
  \usetheme{CambridgeUS}   %[hideothersubsections] beamer Ä£°åµÄģʽ
  % Warsaw or ...,Antibes,PaloAlto,Darmstadt,Frankfurt,Boadilla,Antibes,Luebeck,CambridgeUS,Rochester

%%  With navigation bar: default, boxes, Bergen, Madrid, Pittsburgh, Rochester
%%  With a treelike navigation bar: Antibes, JuanLesPins, Montpellier.
%%  With a TOC sidebar: Berkeley, PaloAlto, Goettingen, Marburg, Hannover
%%  With a mini frame navigation: Berlin, Ilmenau, Dresden, Darmstadt, Frankfurt, Singapore, Szeged
%%  With section and subsection titles: Copenhagen, Luebeck, Malmoe, Warsaw

%% Latex ÉèÖÃ×ÖÌå´óСÃüÁîÓÉСµ½´óÒÀ´ÎΪ£º\tiny \scriptsize \footnotesize \small \normalsize \large \Large \LARGE \huge \Huge
%\definecolor{logo_color}{rgb}{191,22,94}

%  \usefonttheme[onlylarge]{structuresmallcapsserif}%
%  \usefonttheme[onlysmall]{structurebold}%
%  \usefonttheme{serif} % Times New Rome ×ÖÌå
  \usecolortheme{dolphin} % default¶¨Òå×óÉÏÌõ¿ò
  \usecolortheme[RGB={191,22,94}]{structure}
%% Inner color themes, ÆäËûÑ¡Ôñ: orchid,albatross,beaver,beetle,default,crane,dolphin,dove,fly,orchid,lily,rose,seagull,seahorse
%% Inner color themes, ÆäËûÑ¡Ôñ: sidebartab,whale,wolverine

  \useinnertheme{rectangles} % default,circles,margin,rounded,rectangles
%%  \useoutertheme{split} % default£¬infolines£¬miniframes,shadow,smoothbars,smoothtree,tree,sidebar

%  \useoutertheme[height=0.1\textwidth,width=0.15\textwidth,hideothersubsections]{sidebar}
  \setbeamercovered{transparent} % dynamic,transparent,invisible
  % or whatever (possibly just delete it)
}

%%%%%%%%%%%%%%%%%%%%%%%%%%%%%%%%%%%%%%%%%%% ÑÕÉ«¶¨Òå %%%%%%%%%%%%%%%%%%%%%%%%%%%%%%%%%%%%%%%%%%%
%% beamerÖÐÒѾ­¶¨ÒåµÄÑÕÉ«£º
%% red,green,blue,cyan,magenty,yellow,black,darkgray,gray,lightgray,orange,violet,purple,brown

%% ×Ô¶¨ÒåÑÕÉ«£º
%% \xdefinecolor{lanvendar}{rgb}{0.8,0.6,1}
%% \xdefinecolor{olive}{cmyk}{0.64,0,0.95,0.4}
 %\colorlet{structure}{blue!60!black}
% \colorlet{structure}{blue!85!white}      %  ×Ô¶¨ÒåÑÕÉ«,ÓÃ"structure"±íʾ 60%À¶É«+40%ºÚÉ«µÄÑÕÉ«

%\setbeamertemplate{background canvas}[vertical shading][bottom=white,top=structure.fg!25] %%±³¾°É«, ÉÏ25%µÄÀ¶, ¹ý¶Éµ½Ï°×.
%\beamertemplateshadingbackground{white}{blue!25} %ÉèÖý¥±ä(gradient)±³¾°É«,
%\beamersetaveragebackground{yellow!25} % ÉèÖõ¥Ò»µÄ(solid)±³¾°É«
%\beamertemplategridbackground[0.3cm] % ÉèÖÃÕ¤¸ñ(grid ) ±³¾°

%%%%%%%%%%%%%%%%%%%%%%%%%%%%%%%%%%%%%%%%%%% ÏÔʾÉèÖà %%%%%%%%%%%%%%%%%%%%%%%%%%%%%%%%%%%%%%%%%%%
\def\hilite<#1>{\temporal<#1>{\color{white!80!black}}{\color{black}}{\color{white!50!black}}}% magenta
%% ×Ô¶¨ÒåÃüÁî, Ô´×Ô beamer_guide. item Öð²½ÏÔʾʱ, ʹ½«Òª³öÏÖµÄitem¡¢ÕýÔÚÏÔʾµÄitem¡¢ÒѾ­³öÏÖµÄitem¡¢ ³ÊÏÖ²»Í¬ÑÕÉ«.
% \hypersetup{pdfpagemode={FullScreen}} % ĬÈÏÈ«ÆÁ²¥·Å


% Or whatever. Note that the encoding and the font should match. If T1
% does not look nice, try deleting the line with the fontenc.

%%%%%%%%%%%%%%%%%%%%%%%%%%%%%%%%%%%%%%%%%%% Ò³ÃæÉèÖà %%%%%%%%%%%%%%%%%%%%%%%%%%%%%%%%%%%%%%%%%%%
\setbeamertemplate{navigation symbols}{}   %% È¥µôÒ³ÃæÏ·½Ä¬Èϵĵ¼º½Ìõ.
\setcounter{tocdepth}{2} % Ö»Éú³É2¼¶Ä¿Â¼
\setcounter{secnumdepth}{2}
\numberwithin{equation}{section} % ¹«Ê½°´Õ±àºÅ
%\numberwithin{equation}{subsection} % ¹«Ê½°´½Ú±àºÅ
\numberwithin{figure}{section} % ͼƬ°´Õ±àºÅ

\renewcommand{\raggedright}{\leftskip=0pt \rightskip=0pt plus 0cm} %  Á½¶Ë¶ÔÆë
\raggedright

\setbeamertemplate{caption}[numbered] % ͼ±í±àºÅ
\setbeamerfont{caption}{size=\footnotesize} %  ͼ±í±êÌâ×ÖÌå´óСÉèÖÃ

% µ÷ÕûµÚÒ»Ò³±êÌâռλ
% \defbeamertemplate*{frametitle}{smoothbars theme}
%  {
%    \nointerlineskip
%    \begin{beamercolorbox}[wd=\paperwidth,leftskip=.3cm,rightskip=.3cm plus1fil,vmode]{frametitle}
%      \vskip.6ex
%      \usebeamerfont*{frametitle}\insertframetitle%
%      \vskip.6ex
%    \end{beamercolorbox}
%  }

\pgfdeclaremask{cityu_logo}{figure/cityu_logo3}
\pgfdeclareimage[mask=cityu_logo,height=1cm,interpolate=true]{cityu_logo}{figure/cityu_logo3}
\addtobeamertemplate{frametitle}{}{%
%\begin{textblock*}{5cm}(.9\textwidth,-0.85cm)
\begin{textblock*}{5cm}(.88\textwidth,7cm) %%%µ÷½ÚlogoÔÚpptµÄλÖÃ
%\includegraphics[height=1cm,width=1cm]{cityu_logo.pdf}
%bicong%\pgfuseimage{cityu_logo}
\end{textblock*}}

%%%%%%%%%%%%%%%%%%%%%%%%%%%%%%%%%%%%%%%%%%% ×Ô¶¨ÒåÒ³½Å %%%%%%%%%%%%%%%%%%%%%%%%%%%%%%%%%%%%%%%%%%%
\usefoottemplate{\hbox{\tinycolouredline{structure!80!black}{
\color{white}{ \insertshortauthor} \hfill{\insertshortinstitute }
\hfill{\footnotesize \insertframenumber\,/ \inserttotalframenumber}
% \hfill{{\the\year}/{\the\month}/{\the\day}}
}}}

%%%%%%%%%%%%%%%%%%%%%%%%%%%%%%%%%%%%%%%%%%% ÖÐÎÄ×ÖÌå %%%%%%%%%%%%%%%%%%%%%%%%%%%%%%%%%%%%%%%%%%%
\newcommand{\song}{\CJKfamily{song}}
\newcommand{\hei}{\CJKfamily{hei}}
\newcommand{\kai}{\CJKfamily{kai}}
\newcommand{\fs}{\CJKfamily{fs}}

%%%%%%%%%%%%%%%%%%%%%%%%%%%%%%%%%%%%%%%%%%% ÖÐÎĶ¨Àí»·¾³ %%%%%%%%%%%%%%%%%%%%%%%%%%%%%%%%%%%%%%%%%%%
%\newtheorem{theo}{{¶¨Àí}} % \begin{theo}  \end{theo}
%\newtheorem{prop}{{ÃüÌâ}}
%\newtheorem{lem}{{ÒýÀí}}
%\newtheorem{corol}{{ÍÆÂÛ}}[theorem]
%\newtheorem{def}{{¶¨Òå}}
%\newtheorem{exam}{{Àý}}

%%%%%%%%%%%%%%%%%%%%%%%%%%%%% ¶¨ÀíÃüÌâ±àºÅ %%%%%%%%%%%%%%%%%%%%%%%%%%%%%%%%%%
\setbeamertemplate{theorems}[numbered]
%\newtheorem{exam}{Example}[section] % °´Õ±àºÅ
\newboxedtheorem[background=lightgray]{exam}{Example}{section}

\newtheorem{coro}{Corollary}[section]
\newtheorem{thm}{Theorem}[section]
\newtheorem{lem}{Lemma}[section]
\newtheorem{exm}{Example}[section]
\newtheorem{rem}{Remark}[section]
\newtheorem{defi}{Definition}[section]
\newtheorem{prop}{Proposition}[section]

\newtheorem*{refer}{\vspace{-10pt}}
\newtheorem*{con}{Conclusion}
\newtheorem*{fw}{Future Work}
\newtheorem*{spo}{Sponsors}

\newcommand{\EQ}{\begin{eqnarray}}
\newcommand{\EN}{\end{eqnarray}}
\newcommand{\EQQ}{\begin{eqnarray*}}
\newcommand{\ENN}{\end{eqnarray*}}
\newcommand{\BM}{\left[\begin{array}}
\newcommand{\EM}{\end{array}\right]}
\newcommand{\upcite}[1]{$^{\mbox{\scriptsize \cite{#1}}}$}



\newcommand{\nnum}{\nonumber}
\newcommand{\ebox}{\hfill \rule{1.5mm}{1.5mm}}

\def \T{{\mbox{\tiny T}}}
\def \R{{\mbox{\tiny R}}}
\def \RR{{\mathbb{R}}}

\newcommand{\bdefinition}{\begin{defi} \begin{rm} }
\newcommand{\edefinition}{ \end{rm}
\end{defi} }
\newcommand{\bremark}{\begin{rem} \begin{rm} }
\newcommand{\eremark}{ \end{rm}
\end{rem} }
\newcommand{\btheorem}{\begin{thm}  \begin{rm} }
\newcommand{\etheorem}{ \end{rm}
\end{thm} }
\newcommand{\blemma}{\begin{lem} \begin{rm} }
\newcommand{\elemma}{ \end{rm}
\end{lem} }
\newcommand{\bcorollary}{\begin{coro} \begin{rm} }
\newcommand{\ecorollary}{ \end{rm}
\end{coro} }
\newcommand{\bproposition}{\begin{prop} \begin{rm} }
\newcommand{\eproposition}{ \end{rm}
\end{prop} }

\newcommand{\breference}{\begin{refer} \begin{rm} }
\newcommand{\ereference}{ \end{rm}
\end{refer} }

\newcommand{\bconclusion}{\begin{con} \begin{rm} }
\newcommand{\econclusion}{ \end{rm}
\end{con} }

\newcommand{\bfw}{\begin{fw} \begin{rm} }
\newcommand{\efw}{ \end{rm}
\end{fw} }

\newcommand{\bspo}{\begin{spo} \begin{rm} }
\newcommand{\espo}{ \end{rm}
\end{spo} } 